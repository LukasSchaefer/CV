% inspiration: https://github.com/sebastianruder/cv, https://github.com/deedy/Deedy-Resume

\documentclass[]{lukas-cv-openfont}
\usepackage[utf8]{inputenc}
\usepackage{textcomp}
\usepackage{tgpagella}
\usepackage{latexsym}
\usepackage{amssymb}
\usepackage{multicol}
\usepackage{ragged2e}
\pagestyle{empty}
\setlength{\tabcolsep}{0em}

% indentsection style, used for sections that aren't already in lists
% that need indentation to the level of all text in the document
\newenvironment{indentsection}[1]%
{\begin{list}{}%
	{\setlength{\leftmargin}{#1}}%
	\item[]%
}
{\end{list}}

% opposite of above; bump a section back toward the left margin
\newenvironment{unindentsection}[1]%
{\begin{list}{}%
	{\setlength{\leftmargin}{-0.5#1}}%
	\item[]%
}
{\end{list}}

% format two pieces of text, one left aligned and one right aligned
\newcommand{\headerrow}[2]
{\begin{tabular*}{\linewidth}{l@{\extracolsep{\fill}}r}
	\fontspec{Helvetica}\fontsize{12pt}{12pt}\selectfont\bfseries{\color{subheadings}#1} &
	\fontspec{Helvetica}\fontsize{12pt}{12pt}\selectfont\bfseries{\color{subheadings}#2} \\
\end{tabular*}}

\newcommand{\locationrow}[2]
{\begin{tabular*}{\linewidth}{l@{\extracolsep{\fill}}r}
        \color{headings}\scshape\fontspec{Heiti TC Medium}\fontsize{10pt}{12pt}\selectfont{#1}  &
        \color{headings}\scshape\fontspec{Heiti TC Medium}\fontsize{10pt}{12pt}\selectfont{#2}  \\
\end{tabular*}}



\begin{document}

\namesectionnorule{Lukas}{Schäfer}{
        \researchgate{Lukas\_Schaefer9}
        \linkedin{lukas-schaefer}
        \github{LukasSchaefer}\\
        \homepage{lukaschaefer.de/profile}
        \email{luki.schaefer96@gmail.com}
        \smartphone{+44 7925 103212}\\
        %\address{}\\
        %\infos{}
}


\sectionTitle{Education}{graduation-cap}
\hrule
\vspace{0.4em}

\noindent
\headerrow{PhD Data Science \& Artificial Intelligence}{12/2019 -- Present}
\\
\locationrow{University of Edinburgh}{Edinburgh, United Kingdom}
\begin{tightitemize}{1.7em}
    \item Principal supervisor: Dr. Stefano V. Albrecht (\href{https://agents.inf.ed.ac.uk}{Autonomous Agents Research Group})
    \item Project: Collaborative Exploration in Multi-Agent Reinforcement Learning using Intrinsic Curiosity
    \item Research: Reinforcement Learning, Multi-Agent Systems, Exploration, Intrinsic Rewards
\end{tightitemize}
\largesectionsep

\noindent
\headerrow{M.Sc. Informatics}{09/2018 -- 08/2019}
\\
\locationrow{University of Edinburgh}{Edinburgh, United Kingdom}
\begin{tightitemize}{1.7em}
    \item Degree classification: \textbf{Distinction} (77.28\%)
    \item MSc thesis: Dissertation: \href{https://www.lukaschaefer.de/assets/files/msc_thesis.pdf}{Curiosity in Multi-Agent Reinforcement Learning (74\%)}
    \item \textbf{DAAD} (German Academic Exchange Service) \textbf{graduate scholarship}
    \item Modules include: Reinforcement Learning, Algorithmic Game Theory and its Applications, Machine Learning and 
    Pattern Recognition, Probabilistic Modelling and Reasoning, Decision Making in Robots and Autonomous Agents
\end{tightitemize}
\largesectionsep

\noindent
\headerrow{B.Sc. Computer Science, minor subject Japanese}{10/2015 -- 09/2018}
\\
\locationrow{Saarland University}{Saarbrücken, Germany}
\begin{tightitemize}{1.7em}
    \item Degree classification: grade of \textbf{1.2} (\href{https://en.wikipedia.org/wiki/Academic_grading_in_Germany}{German scale}) equivalent to UK \textbf{1\textsuperscript{st} class honours}
    \item BSc thesis: \href{https://www.lukaschaefer.de/assets/files/bsc_thesis.pdf}{Domain-Dependent Policy Learning using Neural Networks in Classical Planning} (1.0)
    \item Modules include: Automated Planning, Admissible Search Enhancements, Neural Networks: Implementation and Application, Information Retrieval and Data Mining, Software Engineering, Modern Imperative Programming Languages
\end{tightitemize}
\largesectionsep

\noindent
\headerrow{Abitur 1.0 - Secondary School}{08/2008 -- 06/2015}
\\
\locationrow{Warndtgymnasium Geislautern, Völklingen}{Geislautern, Germany}
%\begin{tightitemize}{1.7em}
%	\item Graduated \textbf{Abitur 1.0}; school year's best student award, computer science and mathematics award of Saarland University
%    %\item Graduated \textbf{Abitur 1.0} with examination subjects:\\
%    %    Mathematics - 15, English - 12, Computer Science - 14, German - 15, History - 15
%    % \item Year's best student award of the Warndtgymnasiums Geislautern
%    % \item Computer science and mathematics award 2015 of Saarland University
%\end{tightitemize}
\largesectionsep


\sectionTitle{Research Experience}{dna}
\hrule
\vspace{0.4em}

\noindent
\headerrow{M.Sc. Dissertation, University of Edinburgh}{05/2019 -- 08/2019}
\\
\locationrow{Autonomous Agents Research Group}{}
\begin{tightitemize}{1.7em}
    \item Applied curiosity as intrinsically computed exploration bonuses for multi-agent reinforcement learning (MARL)
    \item Implemented count- and prediction-based curiosities for value-based and policy-gradient MARL methods using PyTorch
    \item Evaluated the influence of curiosity on cooperative and competitive MARL under partial observability and sparse rewards in a multi-agent particle environment
    \item Applied curiosity led to improved stability and convergence of policy-gradient MARL trained with sparse reward signals
\end{tightitemize}
\largesectionsep

% \noindent
% \headerrow{Reinforcement Learning for Video Game Playing, University of Edinburgh}{09/2018 -- 01/2019}
% \\
% \locationrow{Informatics Research Review}{}
% \begin{tightitemize}{1.7em}
%     \item Reviewed the development of reinforcement learning research for game playing and the challenge from board games
%     Backgammon, Chess and Go to video games focusing on Atari and StarCraft
%     \item Outlined the development of common reinforcement learning approaches and highlighted the challenge of
%     multi-agent tasks, particular in partially-observable environments and proposed recent, promising ideas
% \end{tightitemize}
% \largesectionsep

\noindent
\headerrow{B.Sc. Dissertation, Saarland University}{04/2018 -- 07/2018}
\\
\locationrow{Foundations of Artificial Intelligence (FAI) Group}{}
\begin{tightitemize}{1.7em}
    \item Transferred domain-dependent policy learning Action-Schema Networks to
    classical automated planning
    \item Implemented the network using Keras, slightly adjusted its training for classical planning and extended 
    the \href{http://www.fast-downward.org}{FastDownward planning framework}
    \item Extensive evaluation and analysis was conducted on IPC domains of varying complexity identifying
    limitations in generalisation and scalability
\end{tightitemize}
\largesectionsep

\sectionTitle{Skills}{poll}
\hrule
\ \vspace{-1em}
\begin{multicols}{2}
    \raggedright
    \setlength{\columnsep}{1.5cm}
    \setlength{\columnseprule}{0.2pt}
    \noindent
    \fontspec{Helvetica}\fontsize{12pt}{12pt}\selectfont\bfseries{\color{subheadings}Programming} \vspace{0.3em} \\
    \color{headings}\scshape\fontspec{Heiti TC Medium}\fontsize{10pt}{10pt}\selectfont{Competent}\\
    \fontspec{Heiti TC Medium}\fontsize{10pt}{8pt}\selectfont \color{primary}Python \textbullet{} C++ \textbullet{} SML
    \ \vspace{0.3em} \\
    \color{headings}\scshape\fontspec{Heiti TC Medium}\fontsize{10pt}{10pt}\selectfont{Familiar}\\
    \fontspec{Heiti TC Medium}\fontsize{10pt}{8pt}\selectfont\color{primary}C \textbullet{} Java \textbullet{} Rust \textbullet{} HTML \textbullet{} CSS \textbullet{} Matlab \textbullet{} Bash\\
    \ \\
    \raggedleft
    \fontspec{Helvetica}\fontsize{12pt}{12pt}\selectfont\bfseries{\color{subheadings}Technologies and Tools}\\
    \fontspec{Heiti TC Medium}\fontsize{10pt}{8pt}\selectfont \color{primary}PyTorch \textbullet{} TensorFlow \textbullet{} Keras \textbullet{} NumPy \textbullet{} UNIX \textbullet{} Git\\
    \ \\
    \fontspec{Helvetica}\fontsize{12pt}{12pt}\selectfont\bfseries{\color{subheadings}Languages}\\
    \fontspec{Heiti TC Medium}\fontsize{10pt}{8pt}\selectfont \color{primary}Native in German \textbullet{} Fluent in English \textbullet{} Intermediate in French \textbullet{} Beginner in Japanese
\end{multicols}

\sectionTitle{Teaching Experience}{chalkboard-teacher}
\hrule
\vspace{0.4em}

\noindent
\headerrow{Teaching Assistant, University of Edinburgh}{10/2019 -- Present}
\\
\locationrow{Reinforcement Learning, School of Informatics}{}
\begin{tightitemize}{1.7em}
    \item Designing reinforcement learning (RL) project covering wide range of topics including dynamic programming, single- and multi-agent RL as well as deep RL
    \item Marking project and exam for reinforcement learning course
    \item Advising students on various challenges regarding lecture material and content
\end{tightitemize}
\largesectionsep

\noindent
\headerrow{Voluntary Lecturer and Coach, Saarland University}{09/2017 -- 10/2017}
\\
\locationrow{Mathematics Preparation Course}{}
\begin{tightitemize}{1.7em}
    \item Assisted the organization of the mathematics preparation course for upcoming computer science students% to student life and mathematical concepts covered in their first year
    \item Explained formal languages and predicate logic to $\sim250$ participants in daily lectures of the first week
    \item Supervised two groups to provide feedback and further assistance in daily coaching-sessions
    \item The course received \textbf{BESTE-award} for special student commitment 2017 of Saarland University
\end{tightitemize}
\largesectionsep

\noindent
\headerrow{Teaching Assistant, Saarland University}{10/2016 -- 03/2017}
\\
\locationrow{Programming 1, Dependable Systems and Software Group}{}
\begin{tightitemize}{1.7em}
    \item Taught first-year students fundamental concepts of functional programming, basic complexity theory and inductive correctness proofs in weekly tutorials and office hours
    \item Marked weekly tests as well as mid- and endterm exams
    \item Collectively created learning materials and discussed student progress as part of the whole teaching team
\end{tightitemize}
\largesectionsep


% \sectionTitle{Work Experience}{briefcase}
% \hrule
% \vspace{0.4em}

% \noindent
% \headerrow{Team Advisor, University of Edinburgh}{09/2019 -- Present}
% \\
% \locationrow{HYPED -- University of Edinburgh Hyperloop Team}{}
% \begin{tightitemize}{1.7em}
%     \item Consulting the HYPED software team regarding navigation and sensor filtering to achieve a reliable prototype design competing at the 5th SpaceX Hyperloop Competition
% \end{tightitemize}
% \largesectionsep


\sectionTitle{Project Experience}{code}
\hrule
\vspace{0.4em}

\noindent
\headerrow{Navigation Software Engineer, University of Edinburgh}{09/2018 -- 08/2019}
\\
\locationrow{HYPED -- University of Edinburgh Hyperloop Team}{}
\begin{tightitemize}{1.7em}
    \item Developing navigation system of "The Flying Podsman" Hyperloop prototype using sensor filtering, processing and control techniques to estimate location, orientation and speed of the pod
    \item Finalist for the SpaceX 2019 Hyperloop competition in California in Summer 2019
\end{tightitemize}
\largesectionsep

\noindent
\headerrow{Reinforcement Learning for Soccer Playing, University of Edinburgh}{02/2019 -- 03/2019}
\\
\locationrow{Project for Reinforcement Learning Lecture}{}
\begin{tightitemize}{1.7em}
    \item Implemented several foundational RL methods including value iteration, Q-learning, first-visit Monte Carlo and SARSA for simple control tasks and the \href{https://github.com/LARG/HFO}{half-field-offense (HFO) 2D environment}
    \item Implemented asynchronous 1-step Q-learning with deep Q-networks (DQNs)
    \item Implemented multi-agent RL methods independent Q-learning, joint action learning and WoLF-PHC controlling two cooperating agents in the HFO environment
\end{tightitemize}
\largesectionsep

\noindent
\headerrow{Autonomous Robot Localisation, University of Edinburgh}{09/2018 -- 12/2018}
\\
\locationrow{Group Project for Robotics: Science and Systems Lecture}{}
\begin{tightitemize}{1.7em}
    \item Constructed a four-wheel differential steering mobile robot as group of three for autonomous localisation in a 
    known environment using LEGO aside of technical components including a Raspberry Pi computer
    \item Implemented particle-filter localisation and obstacle avoidance based on IR and sonar sensors
    \item Robot successfully managed to navigate through the constructed arena, detect and communicate points of
    interest using light sensors and return back to its deployment location
\end{tightitemize}
\largesectionsep


% \noindent
% \headerrow{Galaxy-based Search, University of Edinburgh}{09/2018 -- 12/2018}
% \\
% \locationrow{Group Project for Natural Computing Lecture}{}
% \begin{tightitemize}{1.7em}
%     \item Implemented the Galaxy-based Search Algorithm (GbSA) and Particle Swarm Optimisation (PSO) baseline 
%     for PCA approximation as metaheuristic optimisation algorithms
%     \item Evaluated and analysed GbSA and its foundational research paper, outlined limitations, proposed adjustments to 
%     the algorithm and showed their positive impact on performance in an evaluation
% \end{tightitemize}
% \largesectionsep

\noindent
\headerrow{Plagiarism Detection Tool, Saarland University}{04/2017 -- 07/2017}
\\
\locationrow{Group Project for Software Engineering Lecture}{}
\begin{tightitemize}{1.7em}
    %\item Implemented a plagiarism detection tool in Python for Prof. Rossow at Saarland University
    \item Researched, planned and built a reliable similarity detection for text \& code in Python with language-specific analysis for Python and C as a group of five
    \item Designed and implemented a web-based output creation, highlighting similar submissions and plagiarism
    \item Our software is now successfully used in our customer's lectures to detect plagiarism cases on Python code
\end{tightitemize}
\largesectionsep

% \noindent
% \headerrow{Concurrent CDCL SAT-Solver, Saarland University}{07/2017 -- 09/2017}
% \\
% \locationrow{Group Project for Modern Imperative Programming Languages Seminar}{}
% \begin{tightitemize}{1.7em}
%     \item Planned and implemented a concurrent Conflict-Driven Clause Learning SAT-Solver using Rust
%     \item Optimised literal assignment using multiple heuristic strategies, pure variable detection and handling
% \end{tightitemize}
% \largesectionsep

\ \\

[\scshape\fontspec{Heiti TC Medium}\fontsize{10pt}{8pt}\selectfont References available on request - Last updated on \today]
\end{document}
