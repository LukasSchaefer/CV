%%%%%%%%%%%%%%%%%%%%%%%%%%%%%%%%%%%%%%%
% Deedy - One Page Two Column Resume
% LaTeX Template
% Version 1.2 (16/9/2014)
%
% Original author:
% Debarghya Das (http://debarghyadas.com)
%
% Original repository:
% https://github.com/deedydas/Deedy-Resume
%
% IMPORTANT: THIS TEMPLATE NEEDS TO BE COMPILED WITH XeLaTeX
%
% This template uses several fonts not included with Windows/Linux by
% default. If you get compilation errors saying a font is missing, find the line
% on which the font is used and either change it to a font included with your
% operating system or comment the line out to use the default font.
% 

\documentclass[]{deedy-resume}
\usepackage{fancyhdr}
 
\pagestyle{fancy}
\fancyhf{}
 
\rfoot{Page \thepage \hspace{1pt}}
\begin{document}

%%%%%%%%%%%%%%%%%%%%%%%%%%%%%%%%%%%%%%
%
%     LAST UPDATED DATE
%
%%%%%%%%%%%%%%%%%%%%%%%%%%%%%%%%%%%%%%
\lastupdated

%%%%%%%%%%%%%%%%%%%%%%%%%%%%%%%%%%%%%%
%
%     TITLE NAME
%
%%%%%%%%%%%%%%%%%%%%%%%%%%%%%%%%%%%%%%
\namesection{Lukas}{Schäfer}{ \urlstyle{same}
\href{mailto:luki.schaefer96@gmail.com}{luki.schaefer96@gmail.com} | +49 175 2778299 |  \href{https://www.linkedin.com/in/lukas-schaefer}{www.linkedin.com/in/lukas-schaefer}\\
\Large \href{https://lukaschaefer.de/profile}{\textbf{lukaschaefer.de/profile}}\\
}

%%%%%%%%%%%%%%%%%%%%%%%%%%%%%%%%%%%%%%
%
%     COLUMN ONE
%
%%%%%%%%%%%%%%%%%%%%%%%%%%%%%%%%%%%%%%

\begin{minipage}[t]{0.33\textwidth} 

%%%%%%%%%%%%%%%%%%%%%%%%%%%%%%%%%%%%%%
%     EDUCATION
%%%%%%%%%%%%%%%%%%%%%%%%%%%%%%%%%%%%%%

\section{Education} 

\subsection{University of Edinburgh}
\descript{MSc Informatics}
\location{Aug 2019 | Edinburgh, UK}
\location{\textbf{Distinction} (77.28\%)}
Dissertation: Curiosity in Multi-Agent Reinforcement Learning (74\%)\\
DAAD graduate scholarship
\sectionsep

\subsection{Saarland University}
\descript{BSc Computer Science}
\location{Sep 2018 | Saarbrücken, Germany}
\location{Grade \textbf{1.2} (\href{https://en.wikipedia.org/wiki/Academic_grading_in_Germany}{German scale}) - UK 1\textsuperscript{st}}
Dissertation: \href{https://www.lukaschaefer.de/assets/files/thesis.pdf}{Domain-Dependent Policy Learning using Neural Networks in Classical Planning} (1.0)
\sectionsep

\subsection{Warndtgymnasium}
\descript{Abitur | 1.0}
\location{Jun 2015 |  Geislautern, Germany}
\sectionsep

%%%%%%%%%%%%%%%%%%%%%%%%%%%%%%%%%%%%%%
%     COURSEWORK
%%%%%%%%%%%%%%%%%%%%%%%%%%%%%%%%%%%%%%

\section{Coursework}
\subsection{Graduate}
Reinforcement Learning \\
Algorithmic Game Theory and its Applications \\
Machine Learning and Pattern Recognition \\
Probabilistic Modelling and Reasoning \\
Robotics: Science and Systems \\
Decision Making in Robots and Autonomous Agents \\
\sectionsep

\subsection{Undergraduate}
Automated Planning \\
Admissible Search Enhancements \\
Neural Networks: Implementation and Application \\
Information Retrieval and Data Mining \\
Software Engineering \\
\sectionsep

%%%%%%%%%%%%%%%%%%%%%%%%%%%%%%%%%%%%%%
%     SKILLS
%%%%%%%%%%%%%%%%%%%%%%%%%%%%%%%%%%%%%%

\section{Skills}
\subsection{Programming}
\descript{Competent}
Python \textbullet{} C++ \textbullet{} C \textbullet{} Java \textbullet{} SML
\ \vspace{0.3em} \\
\descript{Familiar}
Rust \textbullet{} HTML \textbullet{} CSS \textbullet{} Matlab \textbullet{} Bash \\
\sectionsep

\subsection{Technologies and Tools}
PyTorch \textbullet{} TensorFlow \textbullet{} Keras \textbullet{} NumPy \textbullet{} UNIX \textbullet{} Git \textbullet{} Vim \textbullet{} \LaTeX
% \sectionsep

\ \\
\small \textit{[References available on request]}

%%%%%%%%%%%%%%%%%%%%%%%%%%%%%%%%%%%%%%
%
%     COLUMN TWO
%
%%%%%%%%%%%%%%%%%%%%%%%%%%%%%%%%%%%%%%

\end{minipage} 
\hfill
\begin{minipage}[t]{0.66\textwidth} 


%%%%%%%%%%%%%%%%%%%%%%%%%%%%%%%%%%%%%%
%     RESEARCH
%%%%%%%%%%%%%%%%%%%%%%%%%%%%%%%%%%%%%%

\section{Research}
\runsubsection{MSc Dissertation}
\descript{| Autonomous Agents Research Group}
\location{May – Aug 2019 | University of Edinburgh}
%\descript{\href{https://www.lukaschaefer.de/assets/files/thesis.pdf}{``Domain-Dependent Policy Learning using Neural Networks in Classical Planning''}}
\vspace{\topsep}
\begin{tightemize}
    \item Applied curiosity as intrinsically computed exploration bonuses for multi-agent reinforcement learning (MARL)
    \item Implemented count- and prediction-based curiosities to evaluate for value-based and policy-gradient MARL methods using PyTorch
    \item Evaluated and analysed the influence of curiosity on cooperative and competitive MARL involving partial observability and sparse rewards
    \item Applied curiosity led to considerably improved stability and convergence applied to policy-gradient MARL trained with sparse reward signals
\end{tightemize}
\sectionsep

% \runsubsection{BSc Dissertation}
% \descript{| Foundations of Artificial Intelligence Group}
% \location{Apr – Jul 2018 | Saarland University}
% %\descript{\href{https://www.lukaschaefer.de/assets/files/thesis.pdf}{``Domain-Dependent Policy Learning using Neural Networks in Classical Planning''}}
% \vspace{\topsep} % Hacky fix for awkward extra vertical space
% \begin{tightemize}
%     \item Transferred domain-dependent policy learning neural network architecture of Action-Schema Networks to
%     classical automated planning
%     \item Implemented the network using Keras, slightly adjusted its training for classical planning and extended 
%     the FastDownward planning framework
%     \item Extensive evaluation and analysis was conducted on IPC domains of varying complexity identifying
%     limitations in generalisation and scalability
% \end{tightemize}
% \sectionsep


%%%%%%%%%%%%%%%%%%%%%%%%%%%%%%%%%%%%%%
%     WORK EXPERIENCE
%%%%%%%%%%%%%%%%%%%%%%%%%%%%%%%%%%%%%%

\section{Work Experience}
\runsubsection{Navigation Team Member}
\descript{| University of Edinburgh Hyperloop Team}
\location{Sep 2018 – Aug 2019 | Edinburgh, UK}
\begin{tightemize}
    \item Developing navigation system of "The Flying Podsman" Hyperloop prototype using sensor filtering, processing and control techniques to estimate location, orientation and speed of the pod
    \item Finalist for the SpaceX 2019 Hyperloop competition in California
\end{tightemize}
\sectionsep


%%%%%%%%%%%%%%%%%%%%%%%%%%%%%%%%%%%%%%
%     Teaching Experience
%%%%%%%%%%%%%%%%%%%%%%%%%%%%%%%%%%%%%%

\section{Teaching Experience}
\runsubsection{Lecturer and Coach}
\descript{| Mathematics Preparation Course}
\location{Sep – Oct 2017 | Saarland University}
\begin{tightemize}
    \item Assisted organisation of preparation course introducing upcoming computer science students to student life and mathematical concepts %covered in their first year
    \item Explained importance of mathematics for CS, formal languages and predicate logic to $\mathbf{\sim 250}$ participants in daily lectures of the first week
    \item Supervised two groups to provide feedback in daily coaching-sessions
    \item The course received \textbf{BESTE-award} for special student commitment 2017 of Saarland University
\end{tightemize}
\sectionsep

\runsubsection{Programming 1 Teaching Assistant}
\descript{| Dependable Systems and Software Group}
\location{Oct 2016 - Mar 2017 | Saarland University}
\begin{tightemize}
    \item Held weekly tutorials and office hours teaching fundamental concepts of functional programming, complexity theory and correctness proofs
    \item Marked weekly tests as well as mid- and endterm exams
    \item Created learning materials and discussed student progress
\end{tightemize}
\sectionsep


%%%%%%%%%%%%%%%%%%%%%%%%%%%%%%%%%%%%%%
%     PROJECT EXPERIENCE
%%%%%%%%%%%%%%%%%%%%%%%%%%%%%%%%%%%%%%

\section{Project Experience}
\runsubsection{Plagiarism Detection Tool}
\descript{| Software Engineering Project}
\location{Apr - Jul 2017 | Saarland University}
\begin{tightemize}
    \item Researched, planned and built a reliable similarity detection for text \& code with language-specific analysis for
    Python and C
    \item Designed and implemented a web-based output creation, highlighting similar submissions and plagiarism cases
    \item Our software is now successfully used in our customer's lectures to detect plagiarism cases on Python code
\end{tightemize}


%%%%%%%%%%%%%%%%%%%%%%%%%%%%%%%%%%%%%%
%     PUBLICATIONS
%%%%%%%%%%%%%%%%%%%%%%%%%%%%%%%%%%%%%%

% \section{Publications} 
% \renewcommand\refname{\vskip -1.5cm} % Couldn't get this working from the .cls file
% \bibliographystyle{abbrv}
% \bibliography{publications}
% \nocite{*}

\end{minipage}
\end{document}  \documentclass[]{article}
