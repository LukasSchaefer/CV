% !TEX program = xelatex
% inspiration: https://github.com/sebastianruder/cv, https://github.com/deedy/Deedy-Resume
\documentclass[]{lukas-cv-openfont}

\usepackage[utf8]{inputenc}
\usepackage{textcomp}
%\usepackage{tgpagella}
\usepackage{latexsym}
\usepackage{amssymb}
\usepackage{multicol}
\usepackage{ragged2e}
\pagestyle{empty}
\setlength{\tabcolsep}{0em}

\urlstyle{same}

% indentsection style, used for sections that aren't already in lists
% that need indentation to the level of all text in the document
\newenvironment{indentsection}[1]%
{\begin{list}{}%
	{\setlength{\leftmargin}{#1}}%
	\item[]%
}
{\end{list}}

% opposite of above; bump a section back toward the left margin
\newenvironment{unindentsection}[1]%
{\begin{list}{}%
	{\setlength{\leftmargin}{-0.5#1}}%
	\item[]%
}
{\end{list}}

% define bibliographies
% \usepackage{splitbib}
% \begin{category}[A]{Conference and Journal Publications}
%     \SBentries{christianos2020shared}
% \end{category}
% \begin{category}[B]{Pre-prints}
%     \SBentries{papoudakis2020comparative}
% \end{category}
\usepackage{bibtopic}
\usepackage[numbers]{natbib}
% set space between entries
\setlength{\bibsep}{3pt} % Space between entries
\bibliographystyle{ieeetr}


\begin{document}
%\expandafter\show\the\font

\namesectionnorule{Lukas}{Schäfer}{
        \homepage{lukaschaefer.com}
        \googlescholar{-yp0O\_IAAAAJ}
        \github{LukasSchaefer}\\
        %\researchgate{Lukas\_Schaefer9}
        \linkedin{lukas-schaefer}
        \email{luki.schaefer96@gmail.com}
        \smartphone{+44 7925 103212}\\
        %\address{}\\
        %\infos{}
}

\vspace{1em}

\sectionTitle{Education}{graduation-cap}

\noindent
\headerrow{PhD Data Science \& Artificial Intelligence}{12/2019 -- Present}
\\
\locationrow{University of Edinburgh}{Edinburgh, United Kingdom}
\begin{tightitemize}{1.7em}
    \item Supervisors: Stefano V. Albrecht (primary) and Amos Storkey (secondary) | Expected graduation: March 2024
    % \item Expected graduation: December 2023
    \item Project: Sample Efficiency and Generalisation in Multi-Agent Reinforcement Learning
    \item Receiving \textbf{Principal's Career Development Scholarship} from the University of Edinburgh
    \item Organisation and hosting of \textbf{RL reading group} with speakers from leading industry (MSR, Google Brain, Deepmind, FAIR) and academic (Oxford University, McGill University, Georgia Institute of Technology, National University of Singapore) labs
    % \item Research Interests: Reinforcement Learning, Multi-Agent Systems%, Generalisation, Exploration, Intrinsic Rewards
\end{tightitemize}
\largesectionsep

\noindent
\headerrow{M.Sc. Informatics}{09/2018 -- 08/2019}
\\
\locationrow{University of Edinburgh}{Edinburgh, United Kingdom}
\begin{tightitemize}{1.7em}
    \item Degree classification: \textbf{Distinction} (77.28\%)
    \item Received \textbf{DAAD} (German Academic Exchange Service) \textbf{graduate scholarship} and \textbf{Stevenson Exchange Scholarship} 
    % \item Modules include: Reinforcement Learning, Algorithmic Game Theory and its Applications, Machine Learning and Pattern Recognition, Probabilistic Modelling and Reasoning, Decision Making in Robots and Autonomous Agents
\end{tightitemize}
\largesectionsep

\noindent
\headerrow{B.Sc. Computer Science, minor subject Japanese}{10/2015 -- 09/2018}
\\
\locationrow{Saarland University}{Saarbrücken, Germany}
\begin{tightitemize}{1.7em}
    \item Degree classification: grade of \textbf{1.2} (\href{https://en.wikipedia.org/wiki/Academic_grading_in_Germany}{German scale}) - within \textbf{top 5\%}% equivalent to UK \textbf{1\textsuperscript{st} class honours} (\textbf{Top 5\%})
    % \item Modules include: Automated Planning, Admissible Search Enhancements, Neural Networks: Implementation and Application, Information Retrieval and Data Mining, Software Engineering, Modern Imperative Programming Languages
\end{tightitemize}
% \largesectionsep

%\noindent
%\headerrow{Abitur - Secondary School}{08/2008 -- 06/2015}
%\\
%\locationrow{Warndtgymnasium Geislautern, Völklingen}{Geislautern, Germany}
%\begin{tightitemize}{1.7em}
%	\item \textbf{Grade of 1.0}; school year's best student award, computer science and mathematics award of Saarland University
%    %\item Graduated \textbf{Abitur 1.0} with examination subjects:\\
%    %    Mathematics - 15, English - 12, Computer Science - 14, German - 15, History - 15
%    % \item Year's best student award of the Warndtgymnasiums Geislautern
%    % \item Computer science and mathematics award 2015 of Saarland University
%\end{tightitemize}
\largesectionsep


\sectionTitle{Publications}{book}
\noindent
\headerrow{Refereed Publications}{}
\vspace{-1.4em}
\begin{btSect}{publications}
    \btPrintNotCited
\end{btSect}
\sectionsep

\noindent
\headerrow{Work in Progress}{}
\vspace{-1.4em}
\begin{btSect}{preprints}
    \btPrintNotCited
\end{btSect}
\sectionsep


\sectionTitle{Experience}{briefcase}

\noindent
\headerrow{Research Intern}{11/2020 -- 03/2021}
\\
\locationrow{Dematic - Technology and Innovation}{Remote}
\begin{tightitemize}{1.7em}
    \item Applying state-of-the-art AI technology to automate large-scale robotic warehouse logistics
\end{tightitemize}
\largesectionsep

\noindent
\headerrow{Navigation Software Engineer and Navigation Advisor}{09/2018 -- 08/2020}
\\
\locationrow{HYPED -- University of Edinburgh Hyperloop Team}{Edinburgh, United Kingdom}
\begin{tightitemize}{1.7em}
    \item Developed navigation system of Hyperloop prototype using Kalman filters, sensor processing and control techniques to estimate location, orientation and speed of the pod
    \item Finalist for the SpaceX 2019 Hyperloop competition in California in Summer 2019
    % \item Advised navigation team on adaptation and implementation of improved sensor and filtering techniques
\end{tightitemize}
% \largesectionsep


\sectionTitle{Skills}{poll}
\vspace{-0.7em}
\begin{multicols}{2}
    \raggedright
    \setlength{\columnsep}{1.5cm}
    \setlength{\columnseprule}{0.2pt}
    \noindent
    \fontspec[Path = ../fonts/Openfonts/lato/]{Lato-Bol}\fontsize{12pt}{12pt}\selectfont\bfseries{\color{subheadings}Programming} \vspace{0.3em} \\
    % \color{headings}\scshape\fontspec[Path = ../fonts/Openfonts/raleway/]{Raleway-Medium}\fontsize{10pt}{10pt}\selectfont{Competent}\\
    \fontspec[Path = ../fonts/Openfonts/raleway/]{Raleway-Medium}\fontsize{10pt}{8pt}\selectfont \color{primary}Python \textbullet{} C++ \textbullet{} SML \textbullet{} Bash\\
    % \ \vspace{0.3em} \\
    % \color{headings}\scshape\fontspec[Path = ../fonts/Openfonts/raleway/]{Raleway-Medium}\fontsize{10pt}{10pt}\selectfont{Familiar}\\
    \ \\
    \fontspec[Path = ../fonts/Openfonts/lato/]{Lato-Bol}\fontsize{12pt}{12pt}\selectfont\bfseries{\color{subheadings}Technologies and Tools}\\
    \fontspec[Path = ../fonts/Openfonts/raleway/]{Raleway-Medium}\fontsize{10pt}{8pt}\selectfont \color{primary}PyTorch \textbullet{} NumPy \textbullet{} UNIX \textbullet{} Git\\

    \raggedleft

    \fontspec[Path = ../fonts/Openfonts/lato/]{Lato-Bol}\fontsize{12pt}{12pt}\selectfont\bfseries{\color{subheadings}Languages}\\
    \fontspec[Path = ../fonts/Openfonts/raleway/]{Raleway-Medium}\fontsize{10pt}{8pt}\selectfont \color{primary}Native in German \textbullet{} Fluent in English \textbullet{} Beginner in Chinese\\ %\textbullet{} Intermediate in French \textbullet{} Beginner in Chinese \textbullet{} Beginner in Japanese
    \ \\
    \fontspec[Path = ../fonts/Openfonts/lato/]{Lato-Bol}\fontsize{12pt}{12pt}\selectfont\bfseries{\color{subheadings}Soft Skills}\\
    \fontspec[Path = ../fonts/Openfonts/raleway/]{Raleway-Medium}\fontsize{10pt}{8pt}\selectfont \color{primary} Teamwork \textbullet{} Teaching \textbullet{} Communication \textbullet{} Organisation\\
\end{multicols}


\sectionTitle{Dissertations}{book-open}

\noindent
\headerrow{M.Sc. Dissertation, Autonomous Agents Research Group}{05/2019 -- 08/2019}
\\
\locationrow{\href{https://www.lukaschaefer.com/assets/files/msc_thesis.pdf}{Curiosity in Multi-Agent Reinforcement Learning (74\%)}}{}
\begin{tightitemize}{1.7em}
    \item Applied count- and prediction-based intrinsic rewards as exploration bonuses to multi-agent reinforcement learning (MARL)
    % \item Implemented count- and prediction-based curiosities for value-based and policy-gradient MARL methods using PyTorch
    \item Evaluated MARL with curiosity under partial observability and sparse rewards in multi-agent particle environments
    \item Proposed multi-agent curiosity led to improved stability and convergence of policy-gradient MARL in sparse-reward tasks
\end{tightitemize}
\largesectionsep

% \noindent
% \headerrow{Reinforcement Learning for Video Game Playing, University of Edinburgh}{09/2018 -- 01/2019}
% \\
% \locationrow{Informatics Research Review}{}
% \begin{tightitemize}{1.7em}
%     \item Reviewed the development of reinforcement learning research for game playing and the challenge from board games
%     Backgammon, Chess and Go to video games focusing on Atari and StarCraft
%     \item Outlined the development of common reinforcement learning approaches and highlighted the challenge of
%     multi-agent tasks, particular in partially-observable environments and proposed recent, promising ideas
% \end{tightitemize}
% \largesectionsep

\noindent
\headerrow{B.Sc. Dissertation, Foundations of Artificial Intelligence (FAI) Group}{04/2018 -- 07/2018}
\\
\locationrow{\href{https://www.lukaschaefer.com/assets/files/bsc_thesis.pdf}{Domain-Dependent Policy Learning using Neural Networks in Classical Planning (1.0)}}{}
\begin{tightitemize}{1.7em}
    \item Transferred policy learning Action-Schema Networks to classical automated planning with adjusted training scheme, Keras implementation and extension of the \href{http://www.fast-downward.org}{FastDownward planning framework}
    \item Extensive evaluation and analysis on IPC domains identifying limitations in generalisation and scalability
\end{tightitemize}
\largesectionsep


\sectionTitle{Teaching Experience}{chalkboard-teacher}

\noindent
\headerrow{Teaching Assistant, University of Edinburgh}{10/2019 -- Present}
\\
\locationrow{Reinforcement Learning, School of Informatics}{}
\begin{tightitemize}{1.7em}
    \item \textbf{Delivering lectures} and \textbf{designing RL coursework} covering wide range of topics from single- to multi-agent and deep RL
    %\item Designing reinforcement learning (RL) project covering wide range of topics including dynamic programming, single- and multi-agent RL as well as deep RL
    \item Marking project and exam for reinforcement learning course
    % \item Advising students on various challenges regarding lecture material and content
\end{tightitemize}
\largesectionsep

\headerrow{M.Sc. Student Supervision, University of Edinburgh}{02/2021 -- 08/2021}
\begin{tightitemize}{1.7em}
    \item Co-supervised two M.Sc. students through project proposal, refinement and execution towards final thesis
    \item Assisted M.Sc. student from their thesis towards a successful workshop submission at NeurIPS 2021
\end{tightitemize}
\largesectionsep

\noindent
\headerrow{Voluntary Lecturer and Coach, Saarland University}{09/2017 -- 10/2017}
\\
\locationrow{Mathematics Preparation Course}{}
\begin{tightitemize}{1.7em}
    % \item Assisted the organization of the mathematics preparation course for upcoming computer science students% to student life and mathematical concepts covered in their first year
    \item Explained formal languages and predicate logic to $\sim250$ participants in daily lectures of the first week
    \item Supervised two groups to provide feedback and further assistance in daily coaching-sessions
    \item The course received \textbf{BESTE-award} for special student commitment 2017 of Saarland University
\end{tightitemize}
\largesectionsep

\noindent
\headerrow{Teaching Assistant, Saarland University}{10/2016 -- 03/2017}
\\
\locationrow{Programming 1, Dependable Systems and Software Group}{}
%\begin{tightitemize}{1.7em}
%    \item Taught first-year students concepts of functional programming, basic complexity theory and inductive correctness proofs in weekly tutorials and office hours
%    %\item Marked weekly tests as well as mid- and endterm exams
%    \item Collectively created learning materials and discussed student progress as part of the whole teaching team
%\end{tightitemize}
\tinysectionsep


% \sectionTitle{Supervision}{user-friends}
% \noindent
% \headerrow{M.Sc. Thesis Supervision, University of Edinburgh}{02/2021 -- 08/2021}
% \begin{tightitemize}{1.7em}
%     \item Co-supervised two M.Sc. students through project proposal, refinement and execution towards final thesis
%     \item Assisted M.Sc. student from their thesis towards a successful workshop submission at NeurIPS 2021
% \end{tightitemize}
% \largesectionsep


\sectionTitle{Reviewing}{search}
\noindent
\begin{tightitemize}{1.0em}
\item \textbf{Conferences}: NeurIPS 2022, ICML 2022, AAMAS 2022, NeurIPS 2021 Datasets and Benchmarks Track
\item \textbf{Workshops}: Pre-Registration Experiment Workshop at NeurIPS 2020
\end{tightitemize}
\largesectionsep


\sectionTitle{Project Experience}{code}

% \noindent
% \headerrow{Navigation Software Engineer, University of Edinburgh}{09/2018 -- 08/2019}
% \\
% \locationrow{HYPED -- University of Edinburgh Hyperloop Team}{}
% \begin{tightitemize}{1.7em}
%     \item Developing navigation system of "The Flying Podsman" Hyperloop prototype using sensor filtering, processing and control techniques to estimate location, orientation and speed of the pod
%     \item Finalist for the SpaceX 2019 Hyperloop competition in California in Summer 2019
% \end{tightitemize}
% \largesectionsep

% \noindent
% \headerrow{Reinforcement Learning for Soccer Playing, University of Edinburgh}{02/2019 -- 03/2019}
% \\
% \locationrow{Project for Reinforcement Learning Lecture}{}
% \begin{tightitemize}{1.7em}
%     \item Implemented several RL methods including value iteration, Q-learning, SARSA and DQN for simple control tasks% and the \href{https://github.com/LARG/HFO}{half-field-offense (HFO) 2D environment}
%     % \item Implemented asynchronous 1-step Q-learning with deep Q-networks (DQNs)
%     % \item Implemented tabular multi-agent RL methods independent Q-learning, joint action learning and WoLF-PHC controlling two cooperating agents in the HFO environment
%     \item Implemented tabular multi-agent RL methods to control two cooperative agents in a 2D football game
% \end{tightitemize}
% \largesectionsep

\noindent
\headerrow{Autonomous Robot Localisation, University of Edinburgh}{09/2018 -- 12/2018}
\\
\locationrow{Group Project for Robotics: Science and Systems Lecture}{}
\begin{tightitemize}{1.7em}
    % \item Constructed a four-wheel differential steering mobile robot as group of three for autonomous localisation in a 
    % known environment using LEGO aside of technical components including a Raspberry Pi computer
    % \item Implemented particle-filter localisation and obstacle avoidance based on IR and sonar sensors
    % \item Robot successfully managed to navigate through the constructed arena, detect and communicate points of interest using light sensors and return back to its deployment location
    \item Constructed a differential steering mobile robot using LEGO, a Raspberry Pi, and an array of IR, camera and sonar sensors
    \item Implemented particle-filter localisation and obstacle avoidance in a predetermined environment
    \item Robot successfully managed to navigate through the constructed arena, detect and communicate points of interest using light sensors and return back to its deployment location
\end{tightitemize}
%\largesectionsep

% \noindent
% \headerrow{Galaxy-based Search, University of Edinburgh}{09/2018 -- 12/2018}
% \\
% \locationrow{Group Project for Natural Computing Lecture}{}
% \begin{tightitemize}{1.7em}
%     \item Implemented the Galaxy-based Search Algorithm (GbSA) and Particle Swarm Optimisation (PSO) baseline 
%     for PCA approximation as metaheuristic optimisation algorithms
%     \item Evaluated and analysed GbSA and its foundational research paper, outlined limitations, proposed adjustments to 
%     the algorithm and showed their positive impact on performance in an evaluation
% \end{tightitemize}
% \largesectionsep

%\noindent
%\headerrow{Plagiarism Detection Tool, Saarland University}{04/2017 -- 07/2017}
%\\
%\locationrow{Group Project for Software Engineering Lecture}{}
%\begin{tightitemize}{1.7em}
%    %\item Implemented a plagiarism detection tool in Python for Prof. Rossow at Saarland University
%    \item Developed a similarity detection tool for submissions in Python with language-specific analysis for Python and C
%    \item Designed and implemented a web-based interface, highlighting similar submissions and presumable cases of plagiarism
%    \item Our software is now successfully used in our customer's lectures to detect plagiarism cases on Python code
%\end{tightitemize}
%\largesectionsep

% \noindent
% \headerrow{Concurrent CDCL SAT-Solver, Saarland University}{07/2017 -- 09/2017}
% \\
% \locationrow{Group Project for Modern Imperative Programming Languages Seminar}{}
% \begin{tightitemize}{1.7em}
%     \item Planned and implemented a concurrent Conflict-Driven Clause Learning SAT-Solver using Rust
%     \item Optimised literal assignment using multiple heuristic strategies, pure variable detection and handling
% \end{tightitemize}
% \largesectionsep

\vspace{-.2em}

\begin{flushleft}
    For more project experience, see \href{https://www.lukaschaefer.com/#projects}{\textbf{lukaschaefer.com/\#projects}}
\end{flushleft}
\sectionsep





\ \\

[\scshape\fontspec[Path = ../fonts/Openfonts/raleway/]{Raleway-Medium}\fontsize{10pt}{8pt}\selectfont References available on request]% - Last updated on \today]
\end{document}
