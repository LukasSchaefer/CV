%!TeX program = xelatex

%-----------------------------------------------------------------------------------------------------------------------------------------------%
%	The MIT License (MIT)
%
%	Copyright (c) 2021 Jitin Nair
%
%	Permission is hereby granted, free of charge, to any person obtaining a copy
%	of this software and associated documentation files (the "Software"), to deal
%	in the Software without restriction, including without limitation the rights
%	to use, copy, modify, merge, publish, distribute, sublicense, and/or sell
%	copies of the Software, and to permit persons to whom the Software is
%	furnished to do so, subject to the following conditions:
%	
%	THE SOFTWARE IS PROVIDED "AS IS", WITHOUT WARRANTY OF ANY KIND, EXPRESS OR
%	IMPLIED, INCLUDING BUT NOT LIMITED TO THE WARRANTIES OF MERCHANTABILITY,
%	FITNESS FOR A PARTICULAR PURPOSE AND NONINFRINGEMENT. IN NO EVENT SHALL THE
%	AUTHORS OR COPYRIGHT HOLDERS BE LIABLE FOR ANY CLAIM, DAMAGES OR OTHER
%	LIABILITY, WHETHER IN AN ACTION OF CONTRACT, TORT OR OTHERWISE, ARISING FROM,
%	OUT OF OR IN CONNECTION WITH THE SOFTWARE OR THE USE OR OTHER DEALINGS IN
%	THE SOFTWARE.
%	
%
%-----------------------------------------------------------------------------------------------------------------------------------------------%

%----------------------------------------------------------------------------------------
%	DOCUMENT DEFINITION
%----------------------------------------------------------------------------------------

% article class because we want to fully customize the page and not use a cv template
\documentclass[a4paper,12pt]{article}

%----------------------------------------------------------------------------------------
%	FONT
%----------------------------------------------------------------------------------------

% \usepackage[scaled]{helvet}
% \renewcommand\familydefault{\sfdefault} 
% \usepackage[T1]{fontenc}

\usepackage{fontspec}
\setmainfont{Calibri}
\usepackage{eurosym}

%----------------------------------------------------------------------------------------
%	PACKAGES
%----------------------------------------------------------------------------------------
\usepackage{url}
\usepackage{parskip} 	

%other packages for formatting
\RequirePackage{color}
\RequirePackage{graphicx}
\usepackage[usenames,dvipsnames]{xcolor}
\usepackage[scale=0.9]{geometry}

\usepackage{tabularx}
\usepackage{siunitx}

%for lists within experience section
\usepackage{enumitem}

% centered version of 'X' col. type
\newcolumntype{C}{>{\centering\arraybackslash}X} 

%to prevent spillover of tabular into next pages
\usepackage{supertabular}
\newlength{\fullcollw}
\setlength{\fullcollw}{0.47\textwidth}

%custom \section
\usepackage{titlesec}				
\usepackage{multicol}
\usepackage{multirow}

%CV Sections inspired by: 
%http://stefano.italians.nl/archives/26
\titleformat{\section}{\Large\raggedright}{}{0em}{}[\titlerule]
\titlespacing{\section}{0pt}{10pt}{10pt}

\titleformat{\subsection}{\large\raggedright}{}{0em}{}
\titlespacing{\subsection}{0pt}{7pt}{7pt}

%for publications
\usepackage[
  backend=biber,
  style=numeric, % not ieee
  citestyle=numeric,
  defernumbers=true,
  maxbibnames=99,
  giveninits=true,
  citetracker=true,
  sorting=none,
]{biblatex}
\addbibresource{bibliography.bib}
\defbibfilter{conf_jour}{keyword=conference or keyword=journal}
\setlength\bibitemsep{0.5em}

%Setup hyperref package, and colours for links
\usepackage[unicode, draft=false]{hyperref}
\definecolor{linkcolour}{rgb}{0,0.2,0.6}
\hypersetup{colorlinks,breaklinks,urlcolor=linkcolour,linkcolor=linkcolour}

%for social icons
\usepackage{fontawesome5}
\usepackage{xltxtra,xunicode,academicons}

%debug page outer frames
%\usepackage{showframe}

% job listing environments
\newcommand*{\exptitle}[2]{
    \begin{tabularx}{\linewidth}{@{}l X r@{}}
        \textbf{#1} & \hfill &  #2 \\[5pt]
    \end{tabularx}
}

\newcommand*{\exptitlelong}[4]{
    \begin{tabularx}{\linewidth}{@{}l X r@{}}
        \textbf{#1} & \hfill & #2 \\
        \textbf{#3} & \hfill & \textit{#4} \\[5pt]
    \end{tabularx}
}

% block that is left-aligned and only reaches to a certain width
\newenvironment{expblock}[1][0.75\linewidth]
{
    \begin{minipage}{#1}
}
{
    \end{minipage}
    \expspace
}

\newenvironment{explist}
{
    \begin{itemize}[nosep,after=\strut, leftmargin=1em, itemsep=3pt]%, label=--]
}
{
    \end{itemize}
}


% Social link command
\newcommand*{\homepage}[2]{\href{#1}{\raisebox{-0.05\height}\faGlobe \ #2}}
\newcommand*{\github}[1]{\href{https://www.github.com/#1}{\raisebox{-0.05\height}\faGithub\ #1}}
\newcommand*{\linkedin}[1]{\href{https://www.linkedin.com/in/#1}{\raisebox{-0.05\height}\faLinkedin\ #1}}
\newcommand*{\scholar}[2]{\href{https://scholar.google.com/citations?user=#1}{\raisebox{-0.05\height}\faGraduationCap\ #2}}
\newcommand*{\email}[1]{\href{mailto:#1}{\raisebox{-0.05\height}\faEnvelope \ #1}}
\newcommand*{\phone}[1]{\href{tel:#1}{\raisebox{-0.05\height}\faMobile \ #1}}

% spacing commands
\newcommand*{\expspace}{\vspace{8pt}}


%----------------------------------------------------------------------------------------
%	BEGIN DOCUMENT
%----------------------------------------------------------------------------------------
\begin{document}

% Citations of all publications to show up in bibliography
\nocite{*}

% non-numbered pages
\pagestyle{empty} 

%----------------------------------------------------------------------------------------
%	TITLE
%----------------------------------------------------------------------------------------

% \begin{tabularx}{\linewidth}{ @{}X X@{} }
% \huge{Your Name}\vspace{2pt} & \hfill \emoji{incoming-envelope} email@email.com \\
% \raisebox{-0.05\height}\faGithub\ username \ | \
% \raisebox{-0.00\height}\faLinkedin\ username \ | \ \raisebox{-0.05\height}\faGlobe \ mysite.com  & \hfill \emoji{calling} number
% \end{tabularx}

\begin{tabularx}{\linewidth}{@{} C @{}}
\Huge{Lukas Sch\"afer} \\[7.5pt]
\homepage{https://www.lukaschaefer.com/}{lukaschaefer.com} \ $|$ \
% \github{LukasSchaefer} \ $|$ \
% \linkedin{lukas-schaefer} \ $|$ \
\scholar{-yp0O\_IAAAAJ}{Google Scholar} \ $|$ \
\email{lukas.schaefer@microsoft.com} \\
\end{tabularx}

%Interests/ Keywords/ Summary
\section{Research Profile}
I am a postdoctoral researcher at Microsoft Research, focusing on efficient learning algorithms for decision-making and multi-agent systems. I a co-author of the first introductory textbook on multi-agent reinforcement learning, and have published in top machine learning venues, including NeurIPS, TMLR, and AAMAS. %My research spans the areas of reinforcement learning, multi-agent systems, and imitation learning, with a particular interest in exploration towards sample-efficient learning and generalisation across tasks.

%----------------------------------------------------------------------------------------
%	PUBLICATIONS
%----------------------------------------------------------------------------------------
\section{Publications\hfill {\large Citations on Google Scholar: 1,200+}}

\subsection{Textbook}
\printbibliography[heading=none, type=book, notkeyword=skip]

% \subsection{Under Submission}
% \printbibliography[heading=none, keyword=under_submission, notkeyword=skip]%, env=cvpubs]

\subsection{Journal and Conference Papers (Peer-Reviewed)}
\printbibliography[heading=none, filter=conf_jour, notkeyword=skip]%, env=cvpubs]

\subsection{Workshops (Peer-Reviewed)}
\printbibliography[heading=none, keyword=workshop, notkeyword=skip]%, env=cvpubs]

% \subsection{Preprints}
% \printbibliography[heading=none, keyword=preprint, notkeyword=skip], env=cvpubs]

% \subsection{Theses}
% \printbibliography[heading=none, keyword=thesis, notkeyword=skip]%, env=cvpubs]
\expspace



%----------------------------------------------------------------------------------------
% EXPERIENCE SECTIONS
%----------------------------------------------------------------------------------------

\section{Work Experience -- Current}

\exptitlelong{Postdoctoral Researcher}{Oct 2024 - Present}{Microsoft Research}{Cambridge, UK}
\begin{expblock}
    My postdoctoral research focuses on developing novel imitation learning algorithms that enable decision making from few demonstrations in complex environments. I also contributed to the recording and data pipeline for training \href{https://copilot.microsoft.com/labs/experiments/copilot-gaming-experiences}{real-time world models deployed on Copilot Labs}.\\
    \textbf{Supervisor:} Sergio Valcarcel Macua
\end{expblock}

\section{Work Experience -- Past}

\exptitlelong{Research Intern}{Apr 2023 - Oct 2023}{Microsoft Research}{Cambridge, UK}
\begin{expblock}
    I conducted an empirical study on the effectiveness of 16 visual encoders, including pre-trained vision foundation models, for imitation learning in modern video games. Our findings were presented at the ALA workshop at AAMAS 2024~\cite{schaefer2025visual}.\\
    \textbf{Supervisors:} Sam Devlin and Tabish Rashid
 %    \begin{explist}
	% \item Conducted an empirical study on the effectiveness of various visual encoders, including pre-trained vision foundation models, for imitation learning in modern video games under mentorship of Sam Devlin and Tabish Rashid
	% \item Developed novel platform for imitation learning in Minecraft Dungeons from scratch, including human gameplay recorder, imitation learning framework using PyTorch Lightning, and programmatic interface for online evaluation
 %    \end{explist}
\end{expblock}

\exptitlelong{Research Intern}{Jul 2022 - Dec 2022}{Huawei Noah's Ark Lab}{London, UK}
\begin{expblock}
    I researched ensemble models and how they can guide exploration and improve training stability in multi-agent reinforcement learning. The resulting publication was accepted and presented as an \textbf{oral paper at AAMAS 2025}~\cite{schaefer2025emax}.\\
    \textbf{Supervisor:} David Mguni
 %    \begin{explist}
	% \item Researched the application of ensemble models to guide exploration and improve training stability of multi-agent reinforcement learning under the supervision of David Mguni
	% % \item Researched ensemble value functions as a plug-and-play approach which is shown to improve sample efficiency, policy robustness, and training stability by leveraging aggregation and uncertainty of the value function ensemble
	% % \item Resulting publication was submitted to top-tier conference (under review)
 %    \end{explist}
\end{expblock}

\exptitlelong{Research Intern}{Nov 2020 - Mar 2021}{Dematic - Technology and Innovation}{Remote}
\begin{expblock}
    I designed and implemented an efficient multi-agent robotic warehouse simulator and novel multi-agent reinforcement learning algorithms for scalable robotic warehouse logistics. My internship led to a \textbf{fellowship funded research collaboration}, four further internship projects, and a publication at IROS 2024~\cite{krnjaic2024scalable}.\\
    \textbf{Supervisors:} Aleksandar Krnjaic and Stefano V. Albrecht
 %    \begin{explist}
	% \item Applied multi-agent reinforcement learning to automate robotic warehouse logistics and scale to real-world settings
	% \item My internship led to a \textbf{fellowship funded research collaboration} and four further internship projects
 %    \end{explist}
\end{expblock}

%----------------------------------------------------------------------------------------
%	EDUCATION
%----------------------------------------------------------------------------------------
\section{Education}

\exptitlelong{PhD, Data Science \& Artificial Intelligence}{Dec 2019 - Oct 2024}{University of Edinburgh}{Edinburgh, UK}
\begin{expblock}
    My PhD research focused on novel exploration methods to enable sample-efficient deep reinforcement learning in single-agent and multi-agent settings. I also studied the ability of agents to generalise across tasks, and how meta-learning task representations can facilitate such generalisation.\\
    % \textbf{Thesis:} Efficient Exploration in Single-Agent and Multi-Agent Deep Reinforcement Learning \\
    \textbf{Supervisors:} Stefano V. Albrecht (primary) and Amos Storkey (secondary)
 %    \begin{explist}
	% \item Summary: Researched novel exploration methods for deep reinforcement learning in single-agent and multi-agent settings.
	% % \item \textbf{Thesis:} Efficient Exploration in Single-Agent and Multi-Agent Deep Reinforcement Learning \\
	% \item Supervisors: Stefano V. Albrecht (primary) and Amos Storkey (secondary)
	% % \item \textbf{Thesis:} Efficient Exploration in Single-Agent and Multi-Agent Deep Reinforcement Learning
	% % \item \textbf{Award:} Passed with no corrections
	% % \item \textbf{Supervisors:} Stefano V. Albrecht (primary) and Amos Storkey (secondary)
 %    \end{explist}
\end{expblock}

\exptitlelong{MSc, Informatics}{Sep 2018 - Aug 2019}{University of Edinburgh}{Edinburgh, UK}
\begin{expblock}
    My Master's dissertation project researched the efficacy of curiosity-driven exploration for multi-agent reinforcement learning. I also completed advanced modules on reinforcement learning, machine learning, robotics, and game theory.\\
    % \textbf{Thesis:} Curiosity in Multi-Agent Reinforcement Learning \\
    \textbf{Supervisor:} Stefano V. Albrecht \hspace{1em} | \hspace{1em} 
    \textbf{Award:} Distinction \ (77.28\%)
 %    \begin{explist}
	% \item Grade: Distinction \ (77.28\%)
	% \item Dissertation project: Researched curiosity-driven exploration in multi-agent reinforcement learning.
	% \item Supervisor: Stefano V. Albrecht
	% % \item \textbf{Thesis:} Curiosity in Multi-Agent Reinforcement Learning
	% % \item \textbf{Key modules:} Decision Making in Robots and Autonomous Agents, Robotics: Science and Systems, Reinforcement Learning, Machine Learning and Pattern Recognition, Probabilistic Modelling and Reasoning % Algorithmic Game Theory and its Applications, 
 %    \end{explist}
\end{expblock}

\exptitlelong{BSc, Computer Science}{Oct 2015 - Sep 2018}{Saarland University}{Saarbrücken, Germany}
\begin{expblock}
    My Bachelor's dissertation project extended Action-Schema-Networks to learn heuristic functions for classical planning using neural networks. Beyond foundational modules in mathematics and computer science, I completed advanced modules on automated planning, neural networks, and software engineering.\\
    % \textbf{Thesis:} Domain-Dependent Policy Learning using Neural Networks in Classical PlanningLearning Heuristic Functions for Classical Planning using Neural Networks \\
    \textbf{Supervisor:} J\"org Hoffmann \hspace{1em} | \hspace{1em}
    \textbf{Award:} 1.2 (within top 5\%)
 %    \begin{explist}
	% \item Grade: 1.2 (within top 5\%)
	% \item Dissertation project: Extended Action-Schema-Networks to learn heuristic functions for classical planning using neural networks.
	% \item Supervisor: J\"org Hoffmann
	% % \item \textbf{Modules include:} Automated Planning, Admissible Search Enhancements, Neural Networks: Implementation and Application, Information Retrieval and Data Mining, Software Engineering, Modern Imperative Programming Languages
 %    \end{explist}
\end{expblock}

%----------------------------------------------------------------------------------------
%	TEACHING
%----------------------------------------------------------------------------------------
\section{Teaching Experience}

\exptitle{Textbook Author}{Mar 2022 - Dec 2024}
\begin{expblock}
    Designed and wrote an introductory \textbf{textbook on multi-agent reinforcement learning} with Stefano V. Albrecht and Filippos Christianos (equal contributions). I also co-designed the accompanying \href{https://github.com/marl-book/codebase}{codebase} and developed \href{https://github.com/marl-book/marl-book-exercises}{exercises} for the Barcelona summer school on multi-agent reinforcement learning (2024). % The textbook has been downloaded over X times since its release in ...
\end{expblock}

\exptitlelong{Teaching Assistant}{Oct 2019 - Jun 2022}{University of Edinburgh}{Edinburgh, UK}
\begin{expblock}
    Re-designed the Reinforcement Learning course as teaching assistant for three consecutive years (2019 -- 2022). Delivered lectures and designed coursework on reinforcement learning (including deep and multi-agent RL) for last year undergraduate and M.Sc. students. Supervised and marked coursework and exam scripts for 100+ students.
\end{expblock}

\exptitlelong{Voluntary Lecturer and Coach}{Sep 2017 - Oct 2017}{Saarland University}{Saarbrücken, Germany}
\begin{expblock}
    Delivered daily lectures on formal languages and predicate logic to 250 participants in the mathematics preparation course for upcoming computer science students. The course received \textbf{BESTE-award} for special student commitment 2017 of Saarland University.
\end{expblock}

\exptitlelong{Teaching Assistant}{Oct 2016 - Mar 2017}{Saarland University}{Saarbrücken, Germany}
\begin{expblock}
    Taught functional programming, basic complexity theory, and inductive proofs to first-year undergraduate students in weekly tutorials and office hours. Collectively created learning materials and discussed student progress as part of the whole teaching team. Marked weekly tests, mid-term and final exams.
\end{expblock}

%----------------------------------------------------------------------------------------
%	SUPERVISION
%----------------------------------------------------------------------------------------
\section{Supervision and Mentorship Experience}

% \exptitlelong{Microsoft Research}{Sep 2025 - Present}{Mentorship for Machine Learning}{Cambridge, UK}
% \begin{expblock}
%     Mentoring graduate-level software engineer on a machine learning project leveraging large vision-language models for video understanding. Providing guidance on research methodology, experimental design, and technical implementations.
% \end{expblock}

\exptitlelong{Supervision of PhD Research Intern}{Apr 2025 - Aug 2025}{Microsoft Research}{Cambridge, UK}
\begin{expblock}
    Supervised PhD research intern Somjit Nath (McGill University/ Mila, Canada) during his 4-month internship. His work designed novel data augmentations for efficient imitation learning in video games.%, and led to a submission to a conference.
\end{expblock}

\newpage
\exptitlelong{Supervision of Visiting PhD Student}{May 2022 - May 2023}{University of Edinburgh}{Edinburgh, UK}
\begin{expblock}
    Supervised visiting PhD student Alain Andres Fernandez (Tecnalia, Spain) during his 3-month research visit and following collaboration. The project researched imitation learning for pre-training and concurrent training of reinforcement learning agents, and resulted in a publication at the \textbf{Neurocomputing Journal} \cite{andres2024using}.
\end{expblock}

\exptitlelong{Supervision of Master's Students}{Sep 2021 - May 2022}{University of Edinburgh}{Edinburgh, UK}
\begin{expblock}
    Supervised two Master's students during their M.Sc. dissertation projects:
    \begin{explist}
	\item Rujie Zhong: Data Collection for Policy Evaluation in Reinforcement Learning\\
	Led to publication at \textbf{NeurIPS 2022} \cite{zhong2022datacollection}, after presentation at Workshop on Offline Reinforcement Learning (NeurIPS 2021)
	\item Panagiotis Kyriakou: Reinforcement Learning with Function Approximation in Continuing Tasks: Discounted Return or Average Reward?
    \end{explist}
\end{expblock}

% %----------------------------------------------------------------------------------------
% %	AWARDS
% %----------------------------------------------------------------------------------------
% \section{Awards}
% \begin{tabularx}{\linewidth}{@{}l X@{}}
%     Jun 2025 & \textbf{Best reviewer award} at ICML 2025 \\
%     Jun 2024 & \textbf{Best reviewer award} at ICML 2024 \\
%     Jul 2022 & \textbf{Best reviewer award} at ICML 2022 \\
%     Oct 2017 & \textbf{BESTE-award} for special student commitment 2017 of Saarland University
% \end{tabularx}

%----------------------------------------------------------------------------------------
%	ACADEMIC ENGAGEMENT
%----------------------------------------------------------------------------------------
\section{Research Community Engagement}

\subsection{Reviewing}
\begin{expblock}
    \begin{explist}
	\item 3-time \textbf{best reviewer award} for ICML conference (2022, 2024, 2025)
	\item Journals: Transactions on Machine Learning Research (TMLR, 2024)
	\item Conferences: NeurIPS (2021, 2022, 2023), ICLR (2026), ICML (2021, 2022, 2023, 2024, 2025), AAMAS (2022, 2023, 2024), RLDM (2025), RLC (2024)
	\item Workshops: NeurIPS Pre-registration experiment workshop (2020)
    \end{explist}
\end{expblock}

\subsection{Invited Talks}
\begin{tabularx}{\linewidth}{@{}l X@{}}
    % Dec 2025 & \textbf{ELLIS UnConference, Interactive Learning and Interventional Representations (ILIR) Workshop} \newline Exploiting State and Action Uncertainty for Imitation Learning using Inverse Dynamics Models \\
    Oct 2025 & \textbf{University of Sheffield, ML Seminar Series} \newline Decision-Making in Modern Video Games: From Human Play to World Models \\
    Jul 2025 & \textbf{Belgium-Netherlands Workshop on Reinforcement Learning (BeNeRL)} \newline Decision Making in Video Games \\
    Nov 2024 & \textbf{Gazi University Turkey, AI Research \& Big Data Seminars} \newline An Introduction to the Multi-Agent Reinforcement Learning Textbook \\
    May 2024 & \textbf{Microsoft Research Cambridge} \newline Efficient and Scalable Decision Making In Complex Environments \\
    Mar 2024 & \textbf{University of Maryland, MARL Reading Group} \newline An Introduction to MARL Textbook and EPyMARL Codebase \\
    Feb 2024 & \textbf{Stanford University, Stanford Intelligent Systems Laboratory} \newline Sample-Efficient Multi-Agent Reinforcement Learning \\
    Jul 2022 & \textbf{Berkeley RL Reading Group} \newline Deep Reinforcement Learning for Multi-Agent Interaction \\
\end{tabularx}
\expspace

\subsection{Organisation}

\exptitlelong{Lead Organiser}{Jul 2024 - Mar 2025}{UK Multi-Agent Systems Symposium 2025}{London, UK}
\begin{expblock}
    Co-lead organiser of the \href{https://www.turing.ac.uk/events/uk-multi-agent-systems-symposium-2025-uk-mas}{\textbf{UK Multi-Agent Systems Symposium 2025}} with 200 participants in collaboration with the Alan Turing Institute and King's College London.
\end{expblock}

\exptitlelong{RL Reading Group Organiser}{Sep 2020 - Sep 2022}{University of Edinburgh}{Edinburgh, UK}
\begin{expblock}
    Organised and hosted RL reading group at University of Edinburgh with speakers from industry (e.g.\ DeepMind, MSR, FAIR) and academia (e.g.\ Oxford University, McGill University, NUS)
\end{expblock}

\subsection{Awards}
\exptitlelong{Young Researcher Attendee}{Sep 2022}{Heidelberg Laureate Forum}{Heidelberg, Germany}
\begin{expblock}
    Selected as one of 100 international young researchers in computer science to participate in the prestigious Heidelberg Laureate Forum where I had the opportunity to network and discuss research with laureates of the most prestigious awards in mathematics and computer science.% (Abel Prize, ACM A.M. Turing Award, ACM Prize in Computing, Fields Medal, IMU Abacus Medal and Nevanlinna Prize).
\end{expblock}

\subsection{Open-Source Software Contributions}
\exptitle{\href{https://github.com/uoe-agents/epymarl}{EPyMARL} -- Core Contributor}{}
\begin{expblock}
    Core contributor to the EPyMARL codebase, an open-source codebase for multi-agent reinforcement learning research. EPyMARL has been developed as part of a benchmarking effort~\cite{papoudakis2021benchmarking} and has since received \textbf{over 600 stars on Github}.
\end{expblock}

\exptitle{\href{https://github.com/marl-book/codebase}{MARL Textbook Codebase} -- Core Contributor}{}
\begin{expblock}
    Core contributor to the accompanying codebase to our textbook on multi-agent reinforcement learning~\cite{albrecht2024multi}, an open-source codebase designed for ease of use and teaching in multi-agent reinforcement learning research. Within the first year of its release, the codebase has received \textbf{over 550 stars on Github}.
\end{expblock}

%----------------------------------------------------------------------------------------
%	FUNDING
%----------------------------------------------------------------------------------------
\section{Funding}

% \subsection{Awarded Funding}
\begin{tabularx}{\linewidth}{@{}l l X@{}}
    Dec 2019 - Jun 2024 & \pounds 58,731 & \textbf{Principal's Career Development Scholarship} \newline PhD scholarship from University of Edinburgh \\
    Apr 2023 - Oct 2023 & \pounds 12,860 & \textbf{Microsoft Research internship extension} \newline funding to extend research internship from 3 to 6 months; granted by Xbox after demo of internship project progress\\
    Jul 2022 - Dec 2022 & \pounds 4,160 & \textbf{Huawei Noah's Ark Lab internship extension} \newline funding to extend research internship from 4 to 5 months\\
    Sep 2018 - Aug 2019 & 16,500 \euro & \textbf{DAAD graduate scholarship} \newline postgraduate scholarship from German Academic Exchange Service\\
    Sep 2018 - Aug 2019 & \pounds 300 & \textbf{Stevenson Exchange Scholarship} \newline postgraduate scholarship to support studies in Scotland
\end{tabularx}

% \subsection{Contributions to Funding Awards}
% \exptitle{Industrial Fellowship Funding}{???}
% \begin{expblock}
%     I designed and implemented an efficient multi-agent robotic warehouse simulator and novel multi-agent reinforcement learning algorithms for scalable robotic warehouse logistics. My internship led to a \textbf{fellowship funded research collaboration}, four further internship projects, and a publication at IROS 2024.\\
% \end{expblock}



% %----------------------------------------------------------------------------------------
% %	SKILLS
% %----------------------------------------------------------------------------------------
% \section{Skills}
% \begin{tabularx}{\linewidth}{@{}l X@{}}
%     Some Skills &  \normalsize{This, That, Some of this and that etc.}\\
%     Some More Skills  &  \normalsize{Also some more of this, Some more that, And some of this and that etc.}\\  
% \end{tabularx}

% \vfill
% \center{\footnotesize Last updated: \today}

\end{document}
